% Тут используется класс, установленный на сервере Papeeria. На случай, если
% текст понадобится редактировать где-то в другом месте, рядом лежит файл matmex-diploma-custom.cls
% который в момент своего создания был идентичен классу, установленному на сервере.
% Для того, чтобы им воспользоваться, замените matmex-diploma на matmex-diploma-custom
% Если вы работаете исключительно в Papeeria то мы настоятельно рекомендуем пользоваться
% классом matmex-diploma, поскольку он будет автоматически обновляться по мере внесения корректив
%

% По умолчанию используется шрифт 14 размера. Если нужен 12-й шрифт, уберите опцию [14pt]
\documentclass[12pt]{matmex-diploma}
\usepackage{amsmath}
\usepackage{arcs}
\usepackage{amsfonts}

\usepackage{color} %% это для отображения цвета в коде
\usepackage{listings} %% собственно, это и есть пакет listings

\usepackage{caption}
\DeclareCaptionFont{white}{\color{white}} %% это сделает текст заголовка белым
%% код ниже нарисует серую рамочку вокруг заголовка кода.
\DeclareCaptionFormat{listing}{\colorbox{black}{\parbox{\textwidth}{#1#2#3}}}
\captionsetup[lstlisting]{format=listing,labelfont=white,textfont=white}


%\documentclass[14pt]{matmex-diploma-custom}

\begin{document}
\lstset{ %
language=C++,                 % выбор языка для подсветки (здесь это С)
basicstyle=\small\sffamily, % размер и начертание шрифта для подсветки кода
numbers=right,               % где поставить нумерацию строк (слева\справа)
numberstyle=\tiny,           % размер шрифта для номеров строк
stepnumber=1,                   % размер шага между двумя номерами строк
numbersep=5pt,                % как далеко отстоят номера строк от подсвечиваемого кода
backgroundcolor=\color{white}, % цвет фона подсветки - используем \usepackage{color}
showspaces=false,            % показывать или нет пробелы специальными отступами
showstringspaces=false,      % показывать или нет пробелы в строках
showtabs=false,             % показывать или нет табуляцию в строках
frame=single,              % рисовать рамку вокруг кода
tabsize=2,                 % размер табуляции по умолчанию равен 2 пробелам
captionpos=t,              % позиция заголовка вверху [t] или внизу [b] 
breaklines=true,           % автоматически переносить строки (да\нет)
breakatwhitespace=false, % переносить строки только если есть пробел
escapeinside={\%*}{*)}   % если нужно добавить комментарии в коде
}
% Год, город, название университета и факультета предопределены,
% но можно и поменять.
% Если англоязычная титульная страница не нужна, то ее можно просто удалить.
\filltitle{ru}{
    chair              = {Математико-механический факультет\\ Специальность Астрономия },
    title              = {Вычисление целочисленной неоднозначности в процессе точного местоопределения с помощью MLAMBDA-метода},
    % Здесь указывается тип работы. Возможные значения:
    %   coursework - Курсовая работа
    %   diploma - Диплом специалиста
    %   master - Диплом магистра
    %   bachelor - Диплом бакалавра
    type               = {coursework},
    position           = {студента},
    group              = 491,
    author             = {Стариков Кирилл Игоревич},
    supervisorPosition = {и.о. зав. кафедрой астрономии,\,доцент},
    supervisor         = {Петров С.\,Д.},
    %reviewerPosition   = {ст. преп.},
    %reviewer           = {Привалов А.\,И.},
    %chairHeadPosition  = {д.\,ф.-м.\,н., профессор},
    %chairHead          = {Хунта К.\,Х.},
%   university         = {Санкт-Петербургский Государственный Университет},
%   faculty            = {Математико-механический факультет},
%   city               = {Санкт-Петербург},
%   year               = {2013}
}
\filltitle{en}{
    chair              = {The Faculty of Mathematics and Mechanics \\ Astronomy department},
    title              = {Computing integer ambiguity in process of precise point positioning with MLAMBDA-method},
    author             = {Kirill Starikov},
    supervisorPosition = {docent},
    supervisor         = {Sergei Petrov},
    type               = {coursework},
    %reviewerPosition   = {assistant},
    %reviewer           = {Alexander Privalov},
    %chairHeadPosition  = {professor},
    %chairHead          = {Christobal Junta},
}
\maketitle
\tableofcontents
% У введения нет номера главы
\section*{Введение}
\subsection*{Тема работы}
В данной работе был реализован алгоритм разрешения целочисленной неоднозначности
псевдофазовых измерений, являющихся частью процесса высокоточных абсолютных
местоопределений в ГНСС.
\subsection*{Подход}
Для разрешения целочисленной неоднозначности псевдофазовых измерений был выбран 
метод MLAMBDA~\cite{article:mlambda}, являющийся модификацией давно использующегося
метода LAMBDA~\cite{article:lambda}. Алгоритм, излагающийся в \cite{article:mlambda},
был реализован в моей работе на языке программирования C++. Для работы с матрицами
использовался собственный класс с базовыми матричными операциями.
За опорную точку брались функции, использующиеся в библиотеке RTKLib~\cite{page:rtk}.
\

\section{Обзор метода}
Ререшение целочисленной неоднозначности псефдофазовых измерений является важным
этапом высокоточных местоопределений в ГНСС. Согласно схеме на рис.
~\ref{fig:pppscheme} после фильтрационной процедуры оценивания становятся известны
вектор действительный оценок (float solution) 
\begin{figure}[h]
\centering
\includegraphics{scheme.png}
\caption{Укрупненная блок-схема алгоритма высокоточного местоопределения потребителя с
разрешением целочисленной неоднозначности псевдофазовых измерений (Integer PPP)  \cite{article:ppp}}
\label{fig:pppscheme}
\end{figure}
\newpage
\begin{equation}
    \hat{x} = 
    \begin{bmatrix}
    \hat{P}\\
    \hat{a}
    \label{eq:1}
    \end{bmatrix}
\end{equation} и соответсвующяя ему ковариационная матрица
\begin{equation}
    \hat{Q} = 
    \begin{bmatrix}
    \hat{Q}_{aa}& \hat{Q}_{Pa}\\
    \hat{Q}_{Na}& \hat{Q}_{aa}
    \label{eq:2}
    \end{bmatrix}
\end{equation}
где $\hat{P}$ - вектор действительных оценок всех параметров кроме неоднозначностей, $\hat{a}$ - вектор
действительных оценок целочисленных неоднозначностей $\hat{a}^{j}$, выраженных в циклах. Блок $\hat{Q}_{aa}$ матрицы $\hat{Q}$  (\ref{eq:2}) является ковариационной матрицей вектора $\hat{a} = 
    \begin{bmatrix}
    \hat{a}^{G}_{3}& \hat{a}^{G}_{4}\
    \end{bmatrix}^{-T}$, содержащего действительные значения оценок для целочисленных векторов $\hat{a}^{G}_{i}$ и $\hat{a}^{G}_{j}$, где $i$ и $j$ зависят от ионосферной модели разделённых часов. \\ \\ В соответствии с \cite{article:lambda} осуществляется минимизация в целых числах $N$ следующей квадратичной формы:
    \begin{equation}
        (a-\hat{a})^{T}\hat{Q}_{\hat{a}}^{-1}(a-\hat{a})
        \longrightarrow{\underset{a \ \in \ \mathbb{Z}^n}{min}}
        \label{eq:3}
    \end{equation}
Предполагая, что вектор $a$ может принимать произвольные действительные
значения, квадратичная форма (\ref{eq:3}) в пространстве с координатами $a$ задаѐт
уравнение эллипсоида с центром в точке $\hat{a}$ (вектор $\hat{a}$ содержит
действительные оценки искомых целочисленных неоднозначностей, доступных на выходе
фильтрационной процедуры оценивания, рис. \ref{fig:pppscheme}). Как правило,
ищется некоторое число $k$ целочисленных векторов, $\tilde{a}$, $i = 1, ...,  k$,
задающих последовательно нарастающие значения квадратичной формы (\ref{eq:3}). \\ \\
%    Известно, что на практике матрица $\hat{P}_{NN}$ является плохо обусловленной
%(отношение еѐ максимального и минимального собственных чисел может доходить до
%нескольких десятков тысяч), что порождает сильную вытянутость (или сплюснутость)
%эллипсоида (\ref{eq:3}). С целью повышения эффективности поисковой процедуры минимизации
%квадратичной формы (\ref{eq:3}) применяется линейное целочисленное унимодулярное
%преобразование - LAMBDA-method~\cite{article:mlambda} - идея которого состоит в отображении эллипсоида
%(\ref{eq:3}) в другое целочисленное пространство $M$ такое, что эллипсоид (\ref{eq:3})
%в нѐм преобразуется в фигуру, близкую к шару, которая определяется преобразованной
%квадратичной формой $D^{*}(M)$. Близкая к шару форма преобразованного эллипсоида
%резко сокращает расходы машинного времени на поиск целочисленного минимума
%преобразованной квадратичной формы (\ref{eq:3}). Затем над $k$ найденными целочисленными
%векторами, $\tilde{M}_{i}$, $i=1,...,k$ , доставляющими последовательно
%нарастающие целочисленные минимумы преобразованному эллипсоиду $D^{*}(M)$,
%осуществляется обратное ЦУМП к целочисленному пространству $N$. Результатом разрешения
%целочисленной неоднозначности являются $k$ целочисленных векторов, $\tilde{N}_{i}$, $i=1,...,k$,
%доставляющих последовательно нарастающие целочисленные минимумы $d$, $\tilde{d}_{i}$, $i=1,...,k$
%квадратичной формы (\ref{eq:3}).

\section{LAMBDA-method и MLAMBDA (\cite{article:mlambda})} 

Пусть $\hat{a} \ \in \ \mathbb{R}^n$ - вещественнозначное приближение метода 
наименьших квадратов (МНК) целочисленного вектора парметров $a \ \in \ \mathbb{Z}^n$
(в нашем случае целочисленный вектор неоднозначностей) и $Q_{\hat{a}} \ \in \ \mathbb{R}^{n \times n}$
- симметричная и положительноопределённая ковариационная матрица.
Приближение целочисленного МНК (ILS) $\tilde{a}$ является решением
задачи минимизации (\ref{eq:3}).
Метод LAMBDA состоит из двух этапов - редукция и поиск. Подробно опишем каждый из них.
\subsection{Процесс редукции}
На практике известно \cite{article:ppp}, что матрица $Q_{\hat{a}}^{-1}$ является
плохо обусловленной (отношение еѐ максимального и минимального собственных
чисел может доходить до нескольких десятков тысяч), что порождает сильную
вытянутость эллипсоида (\ref{eq:3}). Для повышения
эффективности поиска минимума квадратичной формы (\ref{eq:3})
применяется линейное целочисленное унимодулярное преобразование.\\
Пусть $Z \ \in \ \mathbb{Z}^{n \times n}$ - унимодулярная матрица, то есть $|det(Z)|=1$.
Очевидно, что $Z^{-1}$ тоже целочисленная. Используются преобразования:
 \begin{equation}
        z = Z^T a, \ \ \hat{z} = Z^T \hat{a}, \ \ Q_{\hat{z}} = Z^T Q_{\hat{z}} Z.
        \label{eq:5}
 \end{equation}
Они приводят (\ref{eq:3}) к задаче следующего вида:
 \begin{equation}
        \underset{z \ \in \ \mathbb{Z}^n}{min} (z - \hat{z})^TQ_{\hat{z}}^{-1}(z - \hat{z})
        \label{eq:6}
 \end{equation}
Этот переход преобразовывает исходный эллипсоид в фигуру, близкую к шару,
которая определяется квадратичной формой (\ref{eq:6}). Близкая к шару форма
преобразованного эллипсоида резко сокращает расходы машинного времени на поиск
целочисленного минимума исходной задачи.\\
Пусть $L^T D L$ факторизация $Q_{\hat{a}}$ и $Q_{\hat{z}}$ имеют 
емеет следующий вид:
 \begin{equation}
        Q_{\hat{a}} = L^T D L; \ \ Q_{\hat{z}} = Z^T L^T D L Z = \tilde{L}^T \tilde{D} \tilde{L}
        \label{eq:7}
 \end{equation}
 Где $L$ и $\tilde{L}$ - нижние унитреугольные матрицы, и $D = diag(d_1,...,d_n)$, 
 $\tilde{D} = diag(\tilde{d}_1,...,\tilde{d}_n)$, $d_i, \tilde{d}_i > 0$. Процесс
 редукции начинается с $L^T D L$ факторизации $Q_{\hat{a}}$ и обновляет элементы
 таким образом, чтобы получить $L^T D L$ факторизацию $Q_{\hat{z}}$. 
 В этом процессе пытаются найти унимодулярную матрицу $Z$ для достижения
 двух целей, которые имеют решающее значение для эффективности процесс поиска:
 (I) $Q_{\hat{z}}$ - диагональна, то есть недиагональные
 элементы $\tilde{L}$ должны быть минимальны;
 (II) Диагональные элементы $\tilde{D}$ распределены в порядке убывания,
 если это возможно, т. е. процесс стремится к выполнению следующего условия:
  \begin{equation}
        \tilde{d}_1 \geq \tilde{d}_2 \geq ... \geq  \tilde{d}_n
        \label{eq:8}
 \end{equation}
%Здесь сделаем замечание. Алгоритм LLL также преследует две вышеупомянутые цели. Фактически,
%Алгоритм редукции LAMBDA основан на идеях Ленстры (Lenstra, 1981), которые были модифицированы
%и приведены к LLL-алгоритму (Hassibi and Boyed, 1998). 
%Подходы, представленные Лиу (Liu et al., 1999) и Щу (Xu, 2001) в основном
%преследуют первую цель, а вторая цель достигается лишь частично.
В методе LAMBDA унимодулярная матрица $Z$ в (\ref{eq:5}) строится последовательностью
целочисленных преобразований Гаусса и перестановок. Преобразования используются
для того, чтобы сделать недиагональные элементы $\tilde{L}$ как можно меньше,
в то время как перестановки устремлют процесс к выполнию условия (\ref{eq:8}).
\subsubsection{Целочисленные преобразования Гаусса}
Целочисленное преобразование Гаусса $Z_{ij}$ имеет следующую форму:
\begin{equation*}
       Z_{ij} = I - \mu e_i e_j^T / ,
\end{equation*}
где $\mu$ - целое число. Заметим, что $Z^{-1}_{ij} = I + \mu e_i e_j^T$.
Применяя $Z_{ij}$ к $L$ справа даёт
\begin{equation*}
       \tilde{L} = L Z_{ij} = L - \mu L e_i e_j^T
\end{equation*}
Таким образом $\tilde{L}$ такая же, как $L$ за исключением
\begin{equation*}
       \tilde{l}_{kj} = l_{kj} - \mu l_{ki}, \ \ k = i, ... , n
\end{equation*}
Чтобы сделать $\tilde{l}_{kj}$ как можно меньше, берут $\mu = \lfloor l_{ij} \rceil $, обеспечивая
\begin{equation}
       |\tilde{l}_{ij}| \leq 1/2, \ \ i > j
       \label{eq:9.1}
\end{equation}
Когда $Z_{ij}$ применяется к $L$ справа, одновременно следует применять
$Z^T_{ij}$ к $\hat{a}$ слева. Все остальные преобразования тоже должны быть посчитаны.
\subsubsection{Перестановки}
Чтобы добиться порядка (\ref{eq:8}), симметричные перестановки ковариационной
матрицы $Q_{\hat{a}}$ требуют процесса редукции. Когда два диагональных элемента
$Q_{\hat{a}}$ меняются местами, факторы $L$ и $D$ факторизации $L^T D L$ должны
быть обновлены. Рассмотрим $L^T D L$-факторизацию матрицы $Q_{\hat{a}}$:
\begin{equation*}
     Q_{\hat{a}} = L^T D L = 
     \begin{bmatrix}
     L^T_{11} & L^T_{21} & L^T_{31} \\
     & L^T_{22} & L^T_{32} \\
     & & L^T_{33}
     \end{bmatrix}
     \begin{bmatrix}
     D_1 & &  \\
     & D_2 & \\
     & & D_3
     \end{bmatrix}
     \begin{bmatrix}
     L_{11} & &  \\
     L_{21} & L_{22} & \\
     L_{31} & L_{32} & L_{33}
     \end{bmatrix}
\end{equation*}
Пусть
\begin{equation*}
     P = 
     \begin{bmatrix}
     0 & 1 \\
     1 & 0
     \end{bmatrix}, \ \
     P_{k,k+1} = 
     \begin{bmatrix}
     I_{k-1} & &  \\
     & P & \\
     & & I{n-k-1}
     \end{bmatrix}
     \label{eq:9}
\end{equation*}
Можно показать, что $P^T_{k,k+1} Q_{\hat{a}} P_{k,k+1}$ имеет $L^T D L$ факторизацию
\begin{equation}
     P^T_{k,k+1} Q_{\hat{a}} P_{k,k+1} = 
     \begin{bmatrix}
     L^T_{11} & \tilde{L}^T_{21} & L^T_{31} \\
     & \tilde{L}^T_{22} & \tilde{L}^T_{32} \\
     & & L^T_{33}
     \end{bmatrix}
     \begin{bmatrix}
     D_1 & &  \\
     & \tilde{D}_2 & \\
     & & D_3
     \end{bmatrix}
     \begin{bmatrix}
     L_{11} & &  \\
     \tilde{L}_{21} & \tilde{L}_{22} & \\
     L_{31} & \tilde{L}_{32} & L_{33}
     \end{bmatrix}
     \label{eq:10}
\end{equation}
где 
\begin{equation}
     \tilde{D}_2 = 
     \begin{bmatrix}
     \tilde{d}_k &  \\
     & \tilde{d}_{k+1}
     \end{bmatrix}, \ \
     \tilde{d}_{k+1} = d_k + l^2_{k+1,k}d_{k+1}, \ \
     \tilde{d}_{k} = \frac{d_k}{\tilde{d}_{k+1}}d_{k+1},
     \label{eq:11}
\end{equation}
\begin{equation}
     \tilde{L}_{22} = 
     \begin{bmatrix}
     1 &  \\
     \tilde{l}_{k+1,k} & 1
     \end{bmatrix}, \ \
     \tilde{l}_{k+1,k} = \frac{d_{k+1}l_{k+1,k}}{\tilde{d}_{k+1}},
     \label{eq:12}
\end{equation}
\begin{equation}
     \tilde{L}_{21} = 
     \begin{bmatrix}
     -l_{k+1,k} & 1 \\
      \frac{d_k}{\tilde{d}_{k+1}}& \tilde{l}_{k+1,k}
     \end{bmatrix}L_{21} =
     \begin{bmatrix}
     -l_{k+1,k} & 1 \\
      \frac{d_k}{\tilde{d}_{k+1}}& \tilde{l}_{k+1,k}
     \end{bmatrix}L(k:k+1,1:k-1),
     \label{eq:13}
\end{equation}
\begin{equation}
     \tilde{L}_{21} = L_{32}P = 
     [L(k+2:n,k+1) \ L(k+2:n,1:k)].
     \label{eq:14}
\end{equation}
Если мы имеем
\begin{equation}
    \tilde{d}_{k+1} < d_{k+1}
    \label{eq:15}
\end{equation}
(из этого следует, что $d_{k} < d_{k+1}$), то перестановки выполнены. Это не 
гарантирует, что $\tilde{d}_{k} \geq \tilde{d}_{k+1}$, но это делает разрыв между
$\tilde{d}_{k}$ и $\tilde{d}_{k+1}$ меньше, чем между $d_{k}$ и $d_{k+1}$. 
Стоит отметить, что между абсолютные значения элементов ниже $l_{k,k}$ и $l_{k+1,k+1}$
в $L$ ограничены выше на 1/2, оценки остаются в силе после перестановки,
за исключением того, что $\tilde{l}_{k+1,k}$ (ур. \ref{eq:12}) может
больше не ограничиваться 1/2. \\ 
Здесь мы хотели бы указать на важное свойство редукции LAMBDA. После завершения
процесса редукции неравенство (\ref{eq:15}) не будет выполняться ни при каких k 
(иначе будет выполнена новая перестановка), таким образом, для $\tilde{L}$ и
$\tilde{D}$, полученных в конце редукции процесса, мы должны иметь (ур. \ref{eq:11})
\begin{equation*}
     \tilde{d}_{k} + \tilde{l}^2_{k+1,k}\tilde{d}_{k+1} \geq \tilde{d}_{k+1}, \ \
     k = 1,2,...,n-1
\end{equation*}
или
\begin{equation}
     \tilde{d}_{k} \geq (1 - \tilde{l}^2_{k+1,k})\tilde{d}_{k+1}, \ \
     k = 1,2,...,n-1
     \label{eq:16}
\end{equation}
Это тот порядок, который может гарантировать LAMBDA редукция, хотя она и
стремится к более сильному порядку (см. ур-ние \ref{eq:8}).
\newpage
\subsection{Модификации процесса редукции}
Рассмотрим теперь какие модификаци процесса редукции были предложенны в \cite{article:mlambda}.
\subsubsection{Стратегия симметричного опорного элемента}
Чтобы добиться неравенства из уравнения (\ref{eq:8}) или из уравнения (\ref{eq:16}),
алгоритм редукции имеет перестановки. В общем случае,
операции перестановок являются самыми дорогостоящими для всего процесса
редукции. Уменьшая количество перестановок, можно ускорить алгоритм.
Для этого в \cite{article:mlambda} было предложено использовать стратегию
симметричного опорного элемента (symmetric pivoting strategy) при вычислении
$L^T D L$-факторизации ковариационной матрицы $Q_{\hat{a}}$  в начале процесса
редукции. Сначала мы рассмотрим вывод алгоритма факторизации $L^T D L$ без опорного
элемента. Предположим, что $Q \ \in \ \mathbb{R}^{n \times n}$ - симметричная
положительноопределённая матрица. Разбиваем $Q = L^T D L$ следующим образом:
\begin{equation*}
     \begin{bmatrix}
     \tilde{Q} & q \\
     q^T & q_{nn}
     \end{bmatrix} = 
     \begin{bmatrix}
     \tilde{L}^{T} & l  \\
     & 1
     \end{bmatrix}
     \begin{bmatrix}
     \tilde{D} &   \\
     & d_n
     \end{bmatrix}
     \begin{bmatrix}
     \tilde{L} &   \\
     l^T & 1
     \end{bmatrix}
\end{equation*}
Поэтому
\begin{equation*}
     d_n = q_{nn}, \ \ l = q/d_n, \ \ \tilde{Q} - ld_nl^T = \tilde{L}^T\tilde{D}\tilde{L}
\end{equation*}
Эти уравнения ясно показывают, как найти $d_n$, $l$, $\tilde{L}$ и $\tilde{D}$. \\
Поскольку мы стремимся к неравенствам в уравнении (\ref{eq:8}) сначала симметрично
переставляем наименьший диагональный элемент матрицы $Q$ на позицию $(n, n)$.
Затем находим $d_n$, $l$ и применяем этот же подход к $\tilde{Q}−l d_n l^T$.
В итоге, мы получим $L^TDL$-факторизацию $Q$ с перестановками. Предположим,
что после первой симметричной перестановки $P_1$ имеем:
\begin{equation*}
    P^T_1QP_1 =
     \begin{bmatrix}
     \tilde{Q} & q \\
     q^T & q_{nn}
     \end{bmatrix}
\end{equation*}
Определим $d_n = q_{nn}$ и $l = q / d_n$. Пусть $\tilde{Q} - ld_nl^T$ имеет
следующую $L^TDL$-факторизацию c симметричной опорой
\begin{equation*}
     \tilde{P}^T(\tilde{Q} - ld_nl^T)\tilde{P} = \tilde{L}^T\tilde{D}\tilde{L}
\end{equation*}
где $\tilde{P}$ - результат перестановки матриц. Тогда легко убедиться, что
\begin{equation*}
    P^TQP = L^TDL, \ \ P = P_1 
    \begin{bmatrix}
    \tilde{P} &   \\
    & 1
    \end{bmatrix}, \ \
    L = \begin{bmatrix}
    \tilde{L} &   \\
    l^T\tilde{P} & 1
    \end{bmatrix}, \ \
    D = \begin{bmatrix}
            \tilde{D} &   \\
            & d_n
        \end{bmatrix},
\end{equation*}
Это и есть $L^TDL$-факторизация $Q$ с симметричным опорным элементом. 
\subsubsection{Стратегия жадного выбора}
После $L^TDL$-факторизации с симметричным опорным элементом следует процесс редукции.
Сокращая количество перестановок, можно сделать её более эффективной.
Как уже изложено выше редукция в методе LAMBDA выполняется последовательно - 
справа-налево. Когда условие $\tilde{d}_{k + 1} < d_{k + 1}$ 
(см. уравнение \ref{eq:15}) выполнено, происходит перестановка пары $(k, k + 1)$,
а затем мы возвращаемся в исходное состояние, т.е. $k = n - 1$. Интуитивно
маловероятно, что такое сокращение будет очень эффективным. Когда мы достигаем
критического индекса $k$, т.е. $d_{k + 1} \ll d_k$ и $\tilde{d}_{k + 1} < d_{k + 1}$,
выполняется перестановка на этой позиции, но вполне вероятно, что мы получим
$d_{k + 2} \ll d_{k + 1}$ и $\tilde{d}_{k + 2} < d_{k + 2}$, поэтому $k + 1$
становится критическим индексом и так далее. Следовательно, перестановки, 
выполненные до того, как мы достигли индекса k, вероятно, будут потрачены впустую. \\
Для решения этой проблемы была предложена стратегия жадного выбора. Вместо цикла
по $k$ от $n − 1$ до $1$, выбирают индекс $k$ так, чтобы $d_{k + 1}$ максимально 
уменьшался при перестановке пары $(k, k + 1)$, т.е. $k$ определяется как
\begin{equation}
     k = arg \underset{1 \leq j \leq n-1}{min} 
     \left\{ \tilde{d}_{j+1} / d_{j+1} : \tilde{d}_{j+1} < d_{j+1} \right\}
     \label{eq:25}
\end{equation}
Если таких $k$ не найдено, то перестановки не выполняются.
\subsubsection{Стратегия ленивых преобразований}
Перестановки в процессе редукции могут изменять позиции недиагональных
элементов $L$. Так, может случиться, что целочисленные преобразования 
Гаусса применяются к одним и тем же элементам $L$ много раз из-за перестановок.
%В частности, если выполняется перестановка для пары $(k, k + 1)$, то изменяется
%$L (k: k + 1, 1: k - 1)$ (см. уравнение \ref{eq:13}) и меняются местами
%два столбца  $L (k + 1: n, k: k + 1)$ (см. уравнение \ref{eq:14}).
%Если до перестановки абсолютные значения элементов $L(k: k + 1, 1: k - 1)$
%ограничены сверху  $1/2$, то после эта граница уже не гарантируются.
%Таким образом, для элементов $L (k: k + 1, 1: k - 1)$, которые теперь больше 
%$1/2$, теряется соответствующее целочисленное преобразование Гаусса, которое как
%раз и ограничило эти элементы сверху на $1/2$ перед этой перестановкой.
%Перестановка также влияет на $l_{k + 1, k}$ (см. уравнение \ref{eq:12}), абсолютное
%значение которого может быть не ограничено сверху $1/2$. Но каждое целочисленное
%преобразование Гаусса, которое применялось для обеспечения $| l_{k + 1, k} | \leq 1/2$
%перед перестановкой, не тратится зря, так как оно необходимо для выполнения этой
%перестановки. Причина в следующем. Заметим, что
%$\tilde{d}_{k + 1} = d_k + l^2_{k + 1, k}d_{k + 1}$ (см. уравнение \ref{eq:11}).
%Цель перестановки - уменьшить $\tilde{d}_{k + 1}$ так, чтобы $l_{k + 1, k}$ была
%как можно меньше, что и реализуется целочисленным преобразованием Гаусса.
Чтобы это избежать, предлагается применять целочисленные преобразования Гаусса
только к некоторым субдиагональным элементам $L$, а затем делать перестановки.
В процессе перестановок целочисленные преобразования Гаусса будут применяться
только к некоторым измененным субдиагональным элементам $L$. Если перестановок
выполнено не будет, то преобразовываются недиагональные элементам $L$.
В частности, в начале процесса редукции, целочисленное преобразование Гаусса
применяется к $l_{k + 1, k}$, когда выполняется следующий критерий:
\begin{equation}
     \text{Критерий 1:} \ \ d_k < d_{k+1}
     \label{eq:26}
\end{equation}
Когда этот критерий не выполнился, это значит, что $d_k$ и $d_{k + 1}$ были в
правильном порядке, поэтому перестановку для пары $(k, k + 1)$ делать не нужно.
Позже целочисленное преобразование Гаусса будет применяется к $l_{k + 1, k}$,
когда выполняется как критерий $1$, так и следующий критерий $2$:
\begin{equation}
     \text{Критерий 2:} \ \ l_{k+1,k} \ \ \text{изменён при последней перестановке}.
     \label{eq:27}
\end{equation}
Этот критерий используется для того, чтобы пропустить неизменнённые субдиагональные
элементы $L$. После перестановки для пары $(k, k + 1)$, три поддиагональных элемента
$L$ меняются. Это: $l_{k, k − 1}, l_{k + 1, k} и l_{k + 2, k + 1}$. Таким образом,
после перестановки, к этим элементам применяется не более трех целочисленных
преобразований Гаусса. Наконец, когда перестановок выполнено не будет,
целочисленное преобразование Гаусса применяется ко всем элементам в строго
нижнетреугольной части $L$.
\subsection{Процесс дискретного поиска}
После процесса редукции запускается процесс дискретного поиска. Для решения
задачи ЦМНК (\ref{eq:6}) эта стратегия используется для перехода 
в пространство $\mathbb{Z}^n$, содержащее решение. Предположим, имеется
следующая оценка целевой функции в уравнении (\ref{eq:6}):
\begin{equation}
        f(z) \overset{def}{=} (z - \hat{z})^TQ_{\hat{z}}^{-1}(z - \hat{z}) \leq \chi^2
        \label{eq:17}
\end{equation}
Решение ищется над эти гиперэллипсоидом. \\ 
Подставляя $L^T D L$ факторизацию $Q_{\hat{z}}$ из (\ref{eq:7}) в (\ref{eq:17}) получим
\begin{equation}
        f(z) = (z - \hat{z})^T\tilde{L}^{-1}\tilde{D}^{-1}\tilde{L}^{-T}(z - \hat{z}) \leq \chi^2
        \label{eq:18}
\end{equation}
Определяя
\begin{equation}
     \tilde{z} = z - \tilde{L}^{-T}(z - \hat{z}),
     \label{eq:19}
\end{equation}
получим 
\begin{equation*}
    \tilde{L}^{-T}(z - \tilde{z}) = z - \hat{z},
\end{equation*}
то есть
\begin{equation}
     \tilde{z}_n = \hat{z}_n, \ \ \
     \tilde{z}_i = \hat{z}_i + \sum^{n}_{j=i+1}{(z_j - \tilde{z}_j)\tilde{l}_{ij}}, \ \ \
     i = n - 1, n - 2, ..., 1
     \label{eq:20}
\end{equation}
Из (\ref{eq:18}) и 
(\ref{eq:19}) получим
\begin{equation*}
        f(z) = (z - \tilde{z})^T\tilde{D}^{-1}(z - \tilde{z}) \leq \chi^2
\end{equation*}
Эвквивалентно можно переписать
\begin{equation}
      f(z) = \frac{(z_1 - \tilde{z}_1)^2}{\tilde{d}_1} +
      \frac{(z_2 - \tilde{z}_2)^2}{\tilde{d}_2} + ... +
      \frac{(z_n - \tilde{z}_n)^2}{\tilde{d}_n} \leq \chi^2
     \label{eq:21}
\end{equation}
Очевидно, что любое $z$, удовлетворяющее этому неравенству, должно также
удовлетворять следующем поэлементным неравенствам:
\newpage
\begin{equation}
     \tilde{z}_n - \tilde{d}^{1/2}_n \chi \leq z_n \leq \tilde{z}_n + \tilde{d}^{1/2}_n,
     \label{eq:22}
\end{equation}
\begin{center}
\text{\vdots}
\end{center}
\begin{equation}
     \tilde{z}_i - \tilde{d}^{1/2}_i \left[{ \chi^2 - \sum^{n}_{j=i+1}{\frac{(z_j - \tilde{z}_j)^2}{\tilde{d}_j}}}\right]^{1/2} \leq
     z_n \leq
     \tilde{z}_i + \tilde{d}^{1/2}_i \left[{ \chi^2 - \sum^{n}_{j=i+1}{\frac{(z_j - \tilde{z}_j)^2}{\tilde{d}_j}}}\right]^{1/2},
     \label{eq:23}
\end{equation}
\begin{center}
\text{\vdots}
\end{center}
\begin{equation}
     \tilde{z}_1 - \tilde{d}^{1/2}_1 \left[{ \chi^2 - \sum^{n}_{j=2}{\frac{(z_j - \tilde{z}_j)^2}{\tilde{d}_j}}}\right]^{1/2} \leq
     z_1 \leq
     \tilde{z}_1 + \tilde{d}^{1/2}_1 \left[{ \chi^2 - \sum^{n}_{j=2}{\frac{(z_j - \tilde{z}_j)^2}{\tilde{d}_j}}}\right]^{1/2},
     \label{eq:24}
\end{equation}
На основе этих соотношений может быть получена процедура поиска. Нижняя и верхняя
границы $z_i$ определяют интервал, который мы называем уровнем $i$.
Целые числа на этом уровне ищутся последовательно от самых маленьких до самых
больших. Каждое возможное целое число на этом уровне будет опробовано, хотя бы
единожды. Как только $z_i$ определяется на уровне $i$, переходят к уровню $i-1$,
чтобы определить $z_{i − 1}$. Если на уровне $i$ не может быть найдено целое число,
происходит возврат на предыдущий уровень $i + 1$, чтобы взять следующее возможное
целое число для $z_{i + 1}$, а затем снова происходит переход на уровень $i$. 
Как только $z_1$ определен на уровне $1$, найден полный целочисленный вектор $z$.
Затем мы начинаем поиск новых целочисленных векторов. Новый процесс начинается с
уровня $1$ для поиска всех остальных возможных целых чисел от наименьшего до
наибольшего. Поиск завершается, когда найденны все возмжные целые числа.\\
Одним из важных вопросов в процессе поиска является задание положительной константы $\chi^2$,
которая определяет размер эллипсоидальной области. В пакете LAMBDA в процессе
поиска размер эллипсоидальной области остается постоянным. Следовательно,
производительность процесса поиска будет сильно зависят от значения $\chi^2$.
Небольшое значение $\chi^2$ может привести к образованию эллипсоидальной области,
не позволяющей минимизизировать исходную задачу (\ref{eq:3}), в то время как 
слишком большое значение может сделать процесс слишком трудоёмким для минимизации.
Пакет LAMBDA устанавливает значение $\chi^2$ следующим образом.
Предположим, что требуется $p$ оптимальных оценок ЦМНК. Если $p \leq n + 1$,
берётся $z_i = \lfloor \tilde{z}_i \rceil $ для $i = n, n − 1,. . . , 1$
(т.е. округление каждого $\tilde{z}_i$ до ближайшего целого) в уравнении 
(\ref{eq:20}), что дает первый целочисленный вектор $z^{(1)}$.
%, который соответствует так называемой точке Бабая в литературе по теории
%информации, см. Бабай (Babai, 1986) и Эйгрелл (Agrell et al., 2002).
Тогда для каждого $i$ $(i = 1, ..., n)$,
полученное $\tilde{z}_i$ округляется до ближайшего целого числа, не изменяя 
остальные элементы $z^{(1)}$, и создавая новый целочисленный вектор. 
На основе этих $n + 1$ целочисленных векторов, $\chi^2$ устанавливается как $p$-ое
наименьшее pешение целевой функции $f (z)$,
что гарантирует хотя бы $p$ кандидатов в эллипсоидальной области.
Если $p > n + 1$, объем эллипсоида устанавливается равным
$p$, а затем вычисляется $\chi^2$.
%В конце раздела сделаем два замечания. Процесс поиска в пакете LAMBDA фактически
%основан на $LDL^T$ факторизации $Q^{-1}_{\hat{a}}$, что вычисляется из $L^TDL$
%факторизации $Q_{\hat{a}}$, то есть нижняя треугольная матрица первого получается 
%инвертированием нижней треугольной матрицы последнего.
%Когда оптимальная оценка $z$, обозначаемая $\check{z}$, найдена, требуется обратное
%преобразование - $\check{a} = Z^{-1} \check{z}$ (см. уравнение \ref{eq:5}).
%Для подробностей об этом вычислении см. De Jonge and Tiberius (1996), Sects. 3.9 и 4.13.
\subsection{Модификации поиска}
%Константа $\chi^2$, играет важную роль в процессе поиска. 
Для поиска (единственной) оптимальной оценки ЦМНК, Тёниссен (Teunissen, 1993) 
предложил использовать стратегию, сужающую область поиска. Как только целочисленный
вектор $z$ в эллипсоидальной области найден, соответствующее значение $f (z)$ в
(\ref{eq:18}) принимается за новое значение для $\chi^2$. Таким образом,
эллипсоидальная область сужается. Как отмечают Де Йонге и Тибериус (De Jonge
and Tiberius, 1996), стратегия сужения может сильно ускоить процесс поиска.
Однако эта стратегия не используется в пакете LAMBDA, который находит несколько
оптимальных оценок ЦМНК. Чтобы сделать процесс поиска более эффективным,
предлагаем расширить стратегию сужения на случай, когда требуется найти более одной
оптимальной оценки. 
%Прежде чем продолжить, мы опишем альтернативу
%(см. Тёниссен (Teunissen, 1995b, Sect. 2.4) и (De Jonge and Tiberius, 1996, Sect. 4.7))
%прямому алгоритму поиска от меньшего к большему на уровне, описанного выше.
Мы ищем целые числа по неубывающему расстоянию к $\tilde{z}_i$ на интервале,
определённом формулой (\ref{eq:23}) на уровне $i$. В частности, если
$\tilde{z}_i \leq \lfloor \tilde{z}_i \rceil$, мы используем следующий порядок поиска:
\begin{equation}
     \lfloor \tilde{z}_i \rceil, \lfloor \tilde{z}_i \rceil - 1,
     \lfloor \tilde{z}_i \rceil + 1, \lfloor \tilde{z}_i \rceil - 2, ... \ ,
     \label{eq:28}
\end{equation}
иначе используем
\begin{equation}
     \lfloor \tilde{z}_i \rceil, \lfloor \tilde{z}_i \rceil + 1,
     \lfloor \tilde{z}_i \rceil - 1, \lfloor \tilde{z}_i \rceil + 2, ... \ .
     \label{eq:29}
\end{equation}
%Эта стратегия поиска была также независимо предложена Шнорром и Эйчнером (1994).
Теперь мы опишем, как применить стратегию сужения к процессу поиска, когда
требуется больше, чем одна оптимальная оценка ЦМНК. Предположим, нам
требуется $p$ оптимальных оценок ЦМНК. Вначале мы устанавливаем $\chi^2$ равным
бесконечности. Очевидно, что первым кандидатом, полученным в процессе поиска
является
\begin{equation}
     z^{(1)} = \left[\lfloor \tilde{z}_1 \rceil, \lfloor \tilde{z}_2 \rceil,
     \lfloor \tilde{z}_3 \rceil, ... \ , \lfloor \tilde{z}_n \rceil\right]^T
     \label{eq:30}
\end{equation}
%Обратите внимание, что $z^{(1)}$ здесь получается процессом поиска,
%а не отдельным процессом, что предлагает пакет LAMBDA (см. параграф 4 предыдущего раздела).
Возьмем второго кандидата $z^{(2)}$, идентичного $z^{(1)}$, за исключением того,
что первый элемент в $z^{(2)}$ - второе ближайшее к $\tilde{z}_i$ целое число.
И $z^{(3)}$ совпадает с $z^{(1)}$, за исключением того, что его первый элемент
принимается как третье ближайшее целое число к $\tilde{z}_i$ и так далее.
Таким образом мы получаем $p$ решений $z^{(1)}$, $z^{(2)}$, · · ·, $z^{(p)}$.
Очевидно, у нас есть $f (z^{(1)}) \leq f (z^{(2)}) ... \leq f (z^{(p)})$ 
(см. уравнение \ref{eq:21}). Затем эллипсоидальная область сужается, задавая
$\chi^2 = f (z^{(p)})$. Это альтернатива методу, используемому методом LAMBDA
для задания $\chi^2$ и его главное преимущество в том, что упрощает определение $\chi^2$.
Также, если $p = 2$, вероятно, значение $\chi^2$, определенное этим методом, меньше,
чем определенное методом LAMBDA, поскольку $d_1$, вероятно, больше, чем другие $d_i$
после процесса редукции (см. уравнение \ref{eq:21}). Затем мы начинаем искать нового
кандидата. Мы возвращаемся на уровень $2$ и берем следующее возможное целое
число для $z_2$. Продолжать процесс поиска будем, пока мы не найдем нового кандидата
на уровне $1$. Теперь мы заменяем кандидата $z^{(j)}$, который удовлетворяет
$f (z^{(i)}) = \chi^2$, новым. Снова сужаем эллипсоидальную область. Следующее $\chi^2$
принимается как $\underset{1 \leq i \leq p}{max}f (z^{(i)})$. Затем мы продолжаем
описанный выше процесс до тех пор, пока не сможем найти новое приближение. 
Так, получаем $p$ оптимальных оценок ЦМНК.

\section{Реализация}
В программе используются три класса: Matrix.h, Lambda.cpp/.h, LambdaTest.cpp/.h. Кратко рассмотрим каждый класс по отдельности.
\subsection{Matrix.h}
Класс Matrix - простой математический матричный класс, написанный на C++ для 
хранения и обработки данных в методах класса LAMBDA. Класс Matrix содержит 
базовые операции, такие как сложение, умножение, доступ к элементам, ввод и вывод, 
создание единичной матрицы и простые методы решения линейных систем. 

\subsection{Lambda.cpp/.h}
Класс Lambda - класс, реализующий алгоритм MLAMBDA~\cite{article:mlambda} - 
линейного целочисленного унимодулярного преобразования неоднозначности методом 
наименьших квадратов. Их математическая трактовка изложена в предыдущей главе. \\ 
Обзор функций класса изложен в Приложении 1.

\subsection{LambdaTest.cpp/.h}
Класс LambdaTest тестирует класс Lambda. Внутри LambdaTest.cpp пользователем 
вводятся вещественнозначный вектор неоднозначности и его ковариационная матрица. 
Реализован вывод результатов на консоль.

% У заключения нет номера главы
\section*{Заключение}
В ходе работы была реализована модификация LAMBDA-метода на языке $C++$. 

\newpage
\section*{Приложения}
\subsection*{Функции класса Lambda.}

\begin{lstlisting}[label=some-code,caption=lambda.h]

Matrix<int> Lambda::computeIntegerSolution (Matrix<double>  &floatAmbiguity, Matrix<double> &ambiguityCovarianceMatrix);

int Lambda::lambda(const int &m, const Matrix<double> &a, Matrix<double> &Q, Matrix<double> &F, Matrix<double> &s);

int validateSolution(const Matrix<double> &S);

int Lambda::search(const int &m, const Matrix<double> &L, const Matrix<double> &D, Matrix<double> &zs, Matrix<double> &zn, Matrix<double> &s);

void  Lambda::reduction(Matrix<double> &L, Matrix<double> &D, Matrix<double> &Z);

void  Lambda::permutations(Matrix<double> &L, Matrix<double> &D, int j, double del, Matrix<double> &Z);

void  Lambda::gaussTransformation(Matrix<double> &L, Matrix<double> &Z, int i, int j); 

int  Lambda::factorization(const Matrix<double> &Q, Matrix<double> &L, Matrix<double> &D);

\end{lstlisting}

\textbf{computeIntegerSolution}(floatAmbiguity, ambiguityCovarianceMatrix) - функция получает
на вход вещественнозначный вектор оценки неоднозначности и его ковариационную
матрицу и возвращаяет вектор целочисленного решения.\\ 

\textbf{lambda}(m, a, Q, F, s) - $m$ - число целочисленных решений, $a$ - 
вещественнозначный вектор оценки неоднозначности, $Q$ - ковариационная матрица
$a$, $F$ - $n \times m$ матрица целочисленных решений, $s$ - сумма квадратов
невязок целочисленных решений $m \times 1$. Этот алгоритм вычисляет матрицу
целочисленных решений $F$ и вектор невязок $s$, используя все описанные ниже методы. \\

\textbf{validateSolution}(S) - функция по сумме квадратов невязок проверяет решение S
по порогу, который задаёт пользователь.  \\

\textbf{search}(m, L, D, zs, zn, s) - $m$ - число целочисленных решений, $L$ и $D$
- части $L^{T}DL$ факторизации ковариационной матрицы $Q_{\hat{a}}$ размера 
$n \times n$, zs - $n \times 1$, zn - $n \times 1$, $s$ - сумма квадратов невязок
целочисленных решений $m \times 1$. Этот алгоритм находит m оптимальных МНК оценок
целочисленного решения. \\
%$Z_b = Z - L^{-T}(Z-Z_s)$\\
%Краткое описание: После процесса редукции запускается процесс дискретного поиска.
%Чтобы решить проблему МНК, стратегия дискретного поиска используется для перевода
%текущего подпространства в пространство целых чисел, которое содержит решение. 
%Применяется процедура сжатие эллипсоида.
%Опишем, как применить стратегию сжатия к процессу поиска, когда требуется более 
%одной оптимальной оценки МНК. Предположим, нам требуется $m$ оптимальных оценок МНК.
%Вначале мы установили $maxDist$ (т.ч. $f(z) = (z - z_s)^TQ^{-1}_{z_s}(z - z_s) \leq maxDist$)
%равным бесконечности. Очевидно, что первым кандидатом, полученным в процессе поиска,
%является окуглённый до целого значения вектор решений $z$. Затем вычисляем второго 
%кандидата равного второму ближайшему целому числу к $z_i$. И так далее. Таким 
%образом мы получаем $m$ кандидатов $z^{(1)}, z^{(2)}, . . ., z^{(m)}$. Очевидно, 
%что $f(z^{(1)}) \leq f (z^{(2)}) \leq ... \leq f(z^{(m)}$). Затем эллипсоидальную 
%область сжимают, полагая $maxDist = f(z^{(m)})$. Затем мы начинаем поиск нового 
%кандидата. Мы возвращаемся на уровень 2 и берем следующее действительное целое 
%число для $z^{(2)}$. Процесс поиска продолжается, пока мы не найдем нового 
%кандидата на уровне 1. Теперь мы заменяем кандидата $z^{(j)}$, который 
%удовлетворяет $f(z^{(j)}) = maxDist$, новым. Снова сжимаем эллипсоидальную область. 
%Более новый $maxDist$ принимается как $max_{1 \leq i \leq m}f(z^{(i)})$. 
%Затем мы продолжаем описанный выше процесс до тех пор, пока не сможем найти 
%нового кандидата. Наконец, мы получаем m оптимальных оценок МНК. \\

\textbf{reduction}(L, D, Z) - LAMBDA-редукция. Пусть $Q_{\hat{a}}$ ковариационная матрица 
вещественной оценки $\hat{a}$ решения размера. Алгоритм вычисляет целочисленную 
униодулярную матрицу $Z$ и $L^{T}DL$ фактроизацию 
$Q_{\hat{z}} = Z^TQ_{\hat{a}}Z = L^{T}DL$, где $L$ и $D$ - части $L^{T}DL$ 
факторизации ковариационной матрицы $Q_{\hat{a}}$ размера $n \times n$, этот 
алгоритм также вычисляет $\hat{z}=Z^T\hat{a}$ \\ 
Краткое описание. Процесс редукции начинается с предпоследнего столбца L и последней пары
диагональных элементов $D$ и пытается достичь первого столбца $L$ и первой пары диагональных элементов D. Когда алгоритм впервые встречает индекс k, алгоритм сначала выполняет
целочисленное преобразование Гаусса на L такое, что абсолютные значения элементов ниже $l_{jj}$
ограничены сверху числом 1/2, и тогда для пары $(k, k + 1)$ происходит перестановка, если для $del = \hat{d}_{j+1} = d_j \ + \ l^2_{j+1,j}d_{j+1}$ выполняется условие, что $del < d_{j+1}$. После перестановки алгоритм перезапускается, т.е. возвращается
обратно в исходное положение. Алгоритм использует переменную $k$ для отслеживания
тех столбцов, чьи недиагональные элементы по величине уже ограничены сверху $1/2$ из-за
предыдущих целочисленных преобразований Гаусса, чтобы в процессе перезапуска никакие новые преобразования для этих столбцов не выполнялись.\\ 

\textbf{permutations}(ML, D, j, del, Z) - функция, реализующая перестановки. 
$L$ и $D$ - части $L^{T}DL$ факторизации матрицы $Q$ размера $n \times n$, $j$ 
- индекс, $del$ - скаляр (из $L^{T}DL$ преобразования 
$del = \hat{d}_{k+1} = d_k \ + \ l^2_{k+1,k}d_{k+1}$), $Z$ - (целочисленная) 
матрица размера $n \times n$.  \\
Краткое описание. Чтобы добиться неубывания элементов матрицы $D$ по возрастанию
индекса, нужны симметричные перестановки матрицы $Q_{\hat{a}}$. Для этого в 
функции переставляются элементы факторизации матрицы $Q_{\hat{a}}$ - $L$ и $D$. 
$eta, \ lam , \ temp1, \ temp2 $ - времнные локальные перемынные для рассчётов. \\ 

\textbf{gaussTransformation}(L, Z, i, j) \- целочисленное преобразование Гаусса.
$L$ - нижняя треугольная матрица размера $n \times n$, \ $(i, \ j)$ - пара индексов, 
$Z$ - (целочисленная) матрица размера $n \times n$. \\ 
Краткое описание. На входе имеется пара индексов. Сперва округляется $l_{ij}$. Далее к $L$ применяется целочисленное преобразование Гаусса $Z_{ij}$, такое что $|(LZ)(i,j)|<1/2$. \\

\textbf{factorization}(Q, L, D) - $L^{T}DL$ факторизация матрицы $Q$ ($Q = L^{T}DL$, 
где $L$ - нижняя треугольная матрица и $D = diag(d_{1}, ..., d_{n}) , \ d_{i} > 0$) \\
Краткое описание. Создаётся копия матрицы $Q$. Сначала диагональные элементы записываются в вектор D. Потом расчитывается нижняя треугольная матрица $L$. \\



\setmonofont[Mapping=tex-text]{CMU Typewriter Text}
\bibliographystyle{ugost2008ls}
\bibliography{diploma.bib}
\end{document}
